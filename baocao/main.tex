% \documentclass{article}

% \documentclass{book}

\documentclass{report} % Chọn cỡ chữ

\usepackage{contents/start/init}

\usepackage{contents/start/vvn}

\begin{document} % Bắt đầu

%%%%%%%%%%%%%%%%%%%%%%%%%%%%%%%%%%

% \input{contents/chapter0/trang_bia}

% \input{contents/chapter0/trang_trang}

% \input{contents/chapter0/trang_bia}

% \input{contents/chapter0/trang_trang}

% \input{contents/chapter0/nhan_xet_cua_giang_vien}

% \includepdf[pages = -]{contents/chapter0/bao_cao_tien_do_1.pdf}

% \includepdf[pages = -]{contents/chapter0/bao_cao_tien_do_2.pdf}

%%%%%%%%%%%%%%%%%%%%%%%%%%%%%%%%%%

\input{contents/chapter0/muc_luc}

%%%%%%%%%%%%%%%%%%%%%%%%%%%%%%%%%%

% \input{contents/chapter0/loi_cam_on}

% \input{contents/chapter0/danh_sach_bang}

% \input{contents/chapter0/danh_sach_hinh_anh}

% \input{contents/chapter0/danh_sach_cac_cum_tu_viet_tat}

% \input{contents/chapter0/danh_sach_cac_thuat_ngu}

\newpage

\chapter*{\centering MỞ ĐẦU}

\addcontentsline{toc}{chapter}{MỞ ĐẦU}

\section*{Lý do chọn đề tài}

Trong quá trình hoạt động kinh doanh, doanh nghiệp có nhu cầu chuyển đổi mô hình kinh doanh linh hoạt để có thể tồn tại và phát triển khi thị trường thay đổi. Từ đó, đáp ứng nhu cầu của khách hàng, mang lại ưu thế cạnh tranh so với các đối thủ. Do đó, các doanh nghiệp cần hệ thống chuyển đổi nhanh chóng để đáp ứng nhu cầu của mô hình kinh doanh và mong đợi của khách hàng.

Trong những năm gần đây, việc áp dụng kiến trúc vi dịch vụ ngày càng phổ biến,  đem lại nhiều lợi ích như  tách các nghiệp vụ kinh doanh   thành các dịch vụ nhỏ độc lập nhau, tắng tính   linh hoạt và khả năng chống chịu sự cố. 



tăng tính mở rộng c 
 
 khả năng mở rộng và dễ dàng phát triển và bảo trì hệ thống.



%%%%%%%%%%%%%%%%%%%%%%%%%%%%%%%%%%

% thiết kế hướng miền

$\Rightarrow$ Kiến trúc vi dịch vụ giải quyết những thách thức và hỗ trợ doanh nghiệp chuyển đổi dễ dàng. Tuy nhiên, để xây dựng được kiến trúc vi dịch vụ tốt, cần phải tạo ra các dịch vụ nhỏ phù hợp và duy trì tính độc lập. Trong đồ án này, em sử dụng thiết kế hướng miền để phân tích và xây dựng kiến trúc vi dịch vụ. Thiết kế hướng miền xác định và tổ chức các dịch vụ dựa trên việc hiểu rõ về lĩnh vực kinh doanh, giúp dự án phản ánh đúng các quy trình và quy tắc kinh doanh.

% Chủ đề đồ án

%%%%%%%%%%%%%%%%%%%%%%%%%%%%%%%%%%%%%!

Nghị định này có hiệu lực thi hành kể từ ngày 01 tháng 7 năm 2022, khuyến khích cơ quan, tổ chức, cá nhân đáp ứng điều kiện về hạ tầng công nghệ thông tin áp dụng quy định về hóa đơn, chứng từ điện tử của Nghị định này trước ngày 01 tháng 7 năm 2022.

%%%%%%%%%%%%%%%%%%%%%%%%%%%%%%%%%%%%%!

$\Rightarrow$ Theo quy định, tất cả các doanh nghiệp, tổ chức và hộ kinh doanh đều bắt buộc phải chuyển từ sử dụng hóa đơn giấy sang hóa đơn điện tử bắt đầu từ tháng 07/2022. Vì vậy, nhu cầu sử dụng và xử lý hóa đơn điện tử trở nên rất lớn. Do đó ở đồ án này, em chọn chủ đề về quản lý hóa đơn điện tử.

Đề tài xxxxxxxxxxx là một xu hướng quan trọng trong phát triển phần mềm và tiềm năng mang lại nhiều lợi ích trong việc cải thiện quá trình quản lý xxxxxxxxxxxxxxxxxxx

% \input{contents/ly_do_chon_de_tai}

\end{document} % Kết thúc

% kết luận, tài liệu tham khảo

%%%%%%%%%%%%%%%%%%%%%%%%%%%%%%%%%%

\section*{Đối tượng và phạm vi nghiên cứu}

\begin{itemize}

\item \textbf{Đối tượng nghiên cứu:}

\item \textbf{Phạm vi nghiên cứu:}

\end{itemize}

\section*{Mot}

% ý nghĩa khoa học và thực tiễn của đề tài.

\subsection*{Đối tượng và phạm vi nghiên cứu}

\begin{itemize}

\item \textbf{Đối tượng nghiên cứu:} Hệ thống quản lí cửa hàng bán hàng.

\item \textbf{Phạm vi nghiên cứu:} Tập trung vào tìm hiểu kiến trúc kiến trúc vi dịch vụ và bước đầu xây dựng kiến trúc kiến trúc vi dịch vụ cho hệ thống quản lí cửa hàng bán hàng, bao gồm các thành phần như quản lý người dùng, quản lí sản phẩm, quản lí đơn hàng.

\end{itemize}

\subsection*{Ý nghĩa thực tiễn}

\begin{itemize}

\item \textbf{Dễ dàng mở rộng:} Kiến trúc kiến trúc vi dịch vụ cho phép mở rộng các phần tử của hệ thống một cách độc lập, giúp cửa hàng bán hàng mở rộng quy mô kinh doanh một cách linh hoạt và nhanh chóng.

\item \textbf{Dễ bảo trì:} Tách biệt các thành phần trong kiến trúc kiến trúc vi dịch vụ giúp giảm thiểu tác động của sự thay đổi lên các thành phần khác, làm cho quá trình bảo trì và cập nhật dễ dàng hơn.

\item \textbf{Tiết kiệm chi phí:} Hiệu quả hơn trong việc sử dụng tài nguyên hệ thống có thể giúp giảm thiểu chi phí vận hành và duy trì hệ thống.

\end{itemize}

%

\section*{Tóm tắt nội dung đồ án}

\addcontentsline{toc}{chapter}{Tóm tắt nội dung đồ án}

Báo cáo đồ án này sẽ được trình bày gồm 3 chương như sau:

\begin{itemize}

\item \textbf{Chương 1: Phân tích thiết kế hệ thống}

\begin{quote}

Nội dung phân tích hệ thống bán hàng.

\end{quote}

\item \textbf{Chương 2: Trình bày về kiến trúc kiến trúc vi dịch vụ}

\begin{quote}

Trình bày các công nghệ, kĩ thuật, nội dung của kiến trúc kiến trúc vi dịch vụ.

\end{quote}

\item \textbf{Chương 3: Các công nghệ đã sử dụng}

\begin{quote}

Trình bày các công nghệ em đã sử dụng trong đồ án này.

\end{quote}

\end{itemize}

Trong quá trình hoàn thành bài báo cáo này không tránh khỏi những thiếu sót. Vì vậy, em mong nhận được sự giúp đỡ và ý kiến đóng góp chân thành từ các thầy cô để em có thể cải thiện và hoàn thiện đề tài này một cách tốt nhất. Sự đóng góp quý báu của các thầy cô sẽ giúp em hiểu sâu hơn về đề tài và nắm vững hơn trong quá trình thực hiện.

\begin{flushleft}

\begin{adjustwidth}{0cm}{0cm}

Em xin chân thành cảm ơn!

\end{adjustwidth}

\begin{adjustwidth}{1cm}{0cm}

Sinh viên,

\end{adjustwidth}

\begin{adjustwidth}{0.8cm}{0cm}

Vũ Văn Nghĩa

\end{adjustwidth}

\end{flushleft}

% %%%%%%%%%%%%%%%%%%%%%%%%%%%%%%%%%%

% \chapter{Giới thiệu}

% \section{Mot}

% \subsection{Mot}

% \subsubsection{Mot}

% \begin{example} Bắt đầu

% \end{example}

%%%%%%%%%%%%%%%%%%%%%%%%%%%%%%%%%%

% Trong quá trình hoạt động kinh doanh, doanh nghiệp có nhu cầu chuyển đổi mô hình kinh doanh linh hoạt để có thể tồn tại và phát triển khi thị trường thay đổi. Từ đó, đáp ứng nhu cầu của khách hàng, mang lại ưu thế cạnh tranh so với các đối thủ. Do đó, các doanh nghiệp cần hệ thống chuyển đổi nhanh chóng để đáp ứng nhu cầu của mô hình kinh doanh và mong đợi của khách hàng.

Trong những năm gần đây, việc áp dụng kiến trúc vi dịch vụ ngày càng phổ biến,  đem lại nhiều lợi ích như  tách các nghiệp vụ kinh doanh   thành các dịch vụ nhỏ độc lập nhau, tắng tính   linh hoạt và khả năng chống chịu sự cố. 



tăng tính mở rộng c 
 
 khả năng mở rộng và dễ dàng phát triển và bảo trì hệ thống.



%%%%%%%%%%%%%%%%%%%%%%%%%%%%%%%%%%

% thiết kế hướng miền

$\Rightarrow$ Kiến trúc vi dịch vụ giải quyết những thách thức và hỗ trợ doanh nghiệp chuyển đổi dễ dàng. Tuy nhiên, để xây dựng được kiến trúc vi dịch vụ tốt, cần phải tạo ra các dịch vụ nhỏ phù hợp và duy trì tính độc lập. Trong đồ án này, em sử dụng thiết kế hướng miền để phân tích và xây dựng kiến trúc vi dịch vụ. Thiết kế hướng miền xác định và tổ chức các dịch vụ dựa trên việc hiểu rõ về lĩnh vực kinh doanh, giúp dự án phản ánh đúng các quy trình và quy tắc kinh doanh.

% Chủ đề đồ án

%%%%%%%%%%%%%%%%%%%%%%%%%%%%%%%%%%%%%!

Nghị định này có hiệu lực thi hành kể từ ngày 01 tháng 7 năm 2022, khuyến khích cơ quan, tổ chức, cá nhân đáp ứng điều kiện về hạ tầng công nghệ thông tin áp dụng quy định về hóa đơn, chứng từ điện tử của Nghị định này trước ngày 01 tháng 7 năm 2022.

%%%%%%%%%%%%%%%%%%%%%%%%%%%%%%%%%%%%%!

$\Rightarrow$ Theo quy định, tất cả các doanh nghiệp, tổ chức và hộ kinh doanh đều bắt buộc phải chuyển từ sử dụng hóa đơn giấy sang hóa đơn điện tử bắt đầu từ tháng 07/2022. Vì vậy, nhu cầu sử dụng và xử lý hóa đơn điện tử trở nên rất lớn. Do đó ở đồ án này, em chọn chủ đề về quản lý hóa đơn điện tử.

Đề tài xxxxxxxxxxx là một xu hướng quan trọng trong phát triển phần mềm và tiềm năng mang lại nhiều lợi ích trong việc cải thiện quá trình quản lý xxxxxxxxxxxxxxxxxxx

% \chapter{Giới thiệu}

% \input{contents/gioi_thieu}

% \section{Giới thiệu về bài toán hóa đơn điện tử}

% \input{contents/gioi_thieu_ve_bai_toan_hoa_don_dien_tu}

% \emph{Theo em tìm hiểu có các khái niệm và căn cứ pháp lý liên quan sau đây:}

% \subsection{Hóa đơn}

% \emph{Theo quy định tại khoản 1 Điều 3 Nghị định 123/2020/NĐ - CP:}

% \input{contents/hoa_don}

% \subsection{Hóa đơn điện tử}

% \emph{Theo quy định tại khoản 2 Điều 3 Nghị định 123/2020/NĐ - CP:}

% \input{contents/hoa_don_dien_tu}

% \subsection{Bắt buộc sử dụng hóa đơn điện tử từ 01/07/2022}

% \emph{Theo quy định tại khoản 1 Điều 59 Nghị định 123/2020/NĐ - CP:}

% \input{contents/bat_buoc_su_dung_hoa_don_dien_tu_tu_01072022}

% \subsection{Lưu trữ hóa đơn điện tử}

% \input{contents/luu_tru_hoa_don_dien_tu}

% \subsection{Một số lợi ích của hóa đơn điện tử}

% \input{contents/mot_so_loi_ich_cua_hoa_don_dien_tu}

%%%%%%%%%%%%%%%%%%%%%%%%%%%%%%%%%%

% \section{Giới thiệu về kiến trúc vi dịch vụ}

% \subsection{Kiến trúc nguyên khối}

% \input{contents/kien_truc_nguyen_khoi}

% \subsection{Kiến trúc vi dịch vụ}

% \input{contents/kien_truc_vi_dich_vu}

% \subsection{Một số đặc điểm và ưu điểm của kiến trúc vi dịch vụ}

% \input{contents/mot_so_dac_diem_va_uu_diem_cua_kien_truc_vi_dich_vu}

% \newpage

% \subsection{Một số nhược điểm và thách thức của kiến trúc vi dịch vụ}

% \input{contents/mot_so_nhuoc_diem_va_thach_thuc_cua_kien_truc_vi_dich_vu}

%%%%%%%%%%%%%%%%%%%%%%%%%%%%%%%%%%

% \section{Giới thiệu về thiết kế hướng miền}

% \input{contents/gioi_thieu_ve_thiet_ke_huong_mien}

%%%%%%%%%%%%%%%%%%%%%%%%%%%%%%%%%%%

%%%%%%%%%%%%%%%%%%%%%%%%%%%%%%%%%% @ \subsection{chưa xong}

% \chapter{Yêu cầu nghiệp vụ}

% \input{contents/yeu_cau_nghiep_vu}

% \subsection {Yêu cầu nghiệp vụ của bài toán phụ}

% \input{contents/yeu_cau_nghiep_vu_cua_bai_toan_phu}

% \subsubsection{Các chức năng tổng quan của bài toán phụ}

% \input{contents/cac_chuc_nang_tong_quan_cua_bai_toan_phu}

% \subsection{Yêu cầu nghiệp vụ chưa xong}

% \input{contents/yeu_cau_nghiep_vu_chua_xong}

%%%%%%%%%%%%%%%%%%%%%%%%%%%%%%%%%% @ \subsection{chưa xong}

%%%%%%%%%%%%%%%%%%%%%%%%%%%%%%%%%% @ \subsection{chưa xong}

%! chưa chắc vì mình có mẫu kt sẽ khác

%! chưa chắc vì mình có mẫu kt sẽ khác

% \chapter{Phân tích thiết kế hệ thống}

% \section{UML Use Case Diagrams}

% \section{UML Activity Diagrams}

% \section{UML Sequence Diagrams}

% \section{UML Class Diagrams}

%%%%%%%%%%%%%%%%%%%%%%%%%%%%%%%%%% @ \subsection{chưa xong}

%%%%%%%%%%%%%%%%%%%%%%%%%%%%%%%%%%

\chapter{Chi tiết và áp dụng thiết kế hướng miền}

\section{Đôi nét về thiết kế hướng miền (DomainDrivenDesign)}

\input{contents/doi_net_ve_thiet_ke_huong_mien}

\input{contents/chuyen_gia_nganh}

\section{Định nghĩa miền (Domain)}

\input{contents/dinh_nghia_mien_domain}

\section{Các mẫu trong thiết kế hướng miền}

\input{contents/cac_khuon_mau_trong_thiet_ke_huong_mien}

%%%%%%%%%%%%%%%%%%%%%%%%%%%%%%%%%%

\section{Các mẫu chiến lược}

\input{contents/gioi_thieu_cac_mau_chien_luoc_strategic}

\newpage

\subsection{Miền phụ (Sub - Domain)}

Một miền lớn được tạo thành từ nhiều \emph{miền phụ (Sub - Domain)}. Trong thực tế, một miền kinh doanh phức tạp không thể có một chuyên gia ngành có kiến thức về tất cả các miền phụ.

\begin{example} Trong miền thương mại điện tử lớn có thể có một số miền phụ sau:

\begin{itemize}

\item \textbf{Miền phụ quản lý hàng tồn kho:} liên quan đến việc quản lý sản phẩm trong kho hàng.

\item \textbf{Miền phụ quản lý khách hàng:} liên quan đến việc quản lý tài khoản khách hàng.

\item \textbf{Miền phụ vận chuyển:} liên quan đến việc quản lý việc vận chuyển giao hàng.

\end{itemize}

\end{example}

% @Slide thì Phân loại các miền phụ

Trong thiết kế hướng miền, có ba loại miền phụ là:

\begin{itemize}

\item Miền phụ chung (Generic Subdomain)

\item Miền phụ cốt lõi (Core Subdomain)

\item Miền phụ hỗ trợ (Supporting Subdomain)

\end{itemize}

\subsubsection{Phân loại các miền phụ}

\input{contents/phan_loai_cac_mien_phu}

\paragraph{Miền phụ chung (Generic Subdomain)}

\input{contents/mien_phu_chung_generic_subdomain}

\paragraph{Miền phụ cốt lõi (Core Subdomain)}

\input{contents/mien_phu_cot_loi_core_subdomain}

\paragraph{Miền phụ hỗ trợ (Supporting Subdomain)}

\input{contents/mien_phu_ho_tro_supporting_subdomain}

\newpage

\subsubsection{Cách xác định các miền phụ}

\input{contents/cach_xac_dinh_cac_mien_phu}

\newpage

\subsubsection{Tại sao cần phân loại các miền phụ?}

\input{contents/tai_sao_can_phan_loai_cac_mien_phu}

\subsubsection{Áp dụng phân loại miền phụ trong đồ án này}

\input{contents/ap_dung_phan_loai_mien_phu_trong_do_an_nay}

\subsection{Mô hình miền (Domain Models)}

\input{contents/mo_hinh_mien_domain_models}

\subsection{Bối cảnh giới hạn (Bounded Context)}

% \input{contents/boi_canh_gioi_han_bounded_context}

\subsection{Ngôn ngữ chung (Ubiquitous Language)}

% \input{contents/ngon_ngu_chung_ubiquitous_language}

\subsection{Bản đồ bối cảnh (Context Maps)}

% \input{contents/ban_do_boi_canh_context_maps}

\subsection{Chi tiết về các mối quan hệ bối cảnh giới hạn}

% \input{contents/chi_tiet_ve_cac_moi_quan_he_boi_canh_gioi_han}

\subsubsection{Mối quan hệ đối xứng (Symmetric Relationship)}

%%%%%%%%%%%%%%%%%%%%%%%%%%%%%%%%%%

\end{document} % Kết thúc

%%%%%%%%%%%%%%%%%%%%%%%%%%%%%%%%%%

%%%%%%%%%%%%%%%%%%%%%%%%%%%%%%%%%%

\subparagraph{Mô hình riêng biệt (Separate Ways)}

\input{contents/mo_hinh_rieng_biet_separate_ways}

\subparagraph{Mô hình hợp tác (Partnership)}

\input{contents/mo_hinh_hop_tac_partnership}

\subparagraph{Mô hình hạt nhân chung (Shared Kernel)}

\input{contents/mo_hinh_hat_nhan_chung_shared_kernel}

\end{document}

\paragraph{Mối quan hệ bất đối xứng (Asymmetric Relationship)}

\input{contents/moi_quan_he_bat_doi_xung_asymmetric_relationship}

\subparagraph{Mô hình khách hàng - nhà cung cấp (Customer - Supplier)}

\input{contents/mo_hinh_khach_hang_nha_cung_cap_customer_supplier}

\subparagraph{Mô hình tuân thủ (Conformist)}

\input{contents/mo_hinh_tuan_thu_conformist}

\subparagraph{Mô hình chống đổ vỡ (Anti Corruption Layer)}

\input{contents/mo_hinh_chong_do_vo_anti_corruption_layer}

\paragraph{Mối quan hệ 1 - nhiều (One to Many Relationship)}

\input{contents/moi_quan_he_1_nhieu_one_to_many_relationship}

\subparagraph{Dịch vụ máy chủ mở (Open Host Service)}

\input{contents/dich_vu_may_chu_mo_open_host_srv}

\subparagraph{Ngôn ngữ được xuất bản (Published Language)}

\input{contents/ngon_ngu_duoc_xuat_ban_published_language}

\subsection{Các mẫu kỹ thuật (Tactical Patterns)}

\end{document}

% Giới thiệu về Tactical Patterns

% \input{contents/cac_mau_ky_thuat_tactical}

\subsubsection{Các đối tượng miền (Domain Object)}

% \input{contents/cac_doi_tuong_mien_domain_object}

\paragraph{Đối tượng thực thể (Entities Objects)}

% \input{contents/doi_tuong_thuc_the_entities_objects}

\paragraph{Đối tượng giá trị (Value Objects)}

% \input{contents/doi_tuong_gia_tri_value_objects}

\paragraph{Miền dịch vụ (Service)}

% \input{contents/mien_dich_vu_srv}

% %! Hướng dẫn 7/4

% %! Hướng dẫn 7/5

% \subsubsection{xxxxxxx}

% Trong quá trình hoạt động kinh doanh, doanh nghiệp có nhu cầu chuyển đổi mô hình kinh doanh linh hoạt để có thể tồn tại và phát triển khi thị trường thay đổi. Từ đó, đáp ứng nhu cầu của khách hàng, mang lại ưu thế cạnh tranh so với các đối thủ. Do đó, các doanh nghiệp cần hệ thống chuyển đổi nhanh chóng để đáp ứng nhu cầu của mô hình kinh doanh và mong đợi của khách hàng.

Trong những năm gần đây, việc áp dụng kiến trúc vi dịch vụ ngày càng phổ biến,  đem lại nhiều lợi ích như  tách các nghiệp vụ kinh doanh   thành các dịch vụ nhỏ độc lập nhau, tắng tính   linh hoạt và khả năng chống chịu sự cố. 



tăng tính mở rộng c 
 
 khả năng mở rộng và dễ dàng phát triển và bảo trì hệ thống.



%%%%%%%%%%%%%%%%%%%%%%%%%%%%%%%%%%

% thiết kế hướng miền

$\Rightarrow$ Kiến trúc vi dịch vụ giải quyết những thách thức và hỗ trợ doanh nghiệp chuyển đổi dễ dàng. Tuy nhiên, để xây dựng được kiến trúc vi dịch vụ tốt, cần phải tạo ra các dịch vụ nhỏ phù hợp và duy trì tính độc lập. Trong đồ án này, em sử dụng thiết kế hướng miền để phân tích và xây dựng kiến trúc vi dịch vụ. Thiết kế hướng miền xác định và tổ chức các dịch vụ dựa trên việc hiểu rõ về lĩnh vực kinh doanh, giúp dự án phản ánh đúng các quy trình và quy tắc kinh doanh.

% Chủ đề đồ án

%%%%%%%%%%%%%%%%%%%%%%%%%%%%%%%%%%%%%!

Nghị định này có hiệu lực thi hành kể từ ngày 01 tháng 7 năm 2022, khuyến khích cơ quan, tổ chức, cá nhân đáp ứng điều kiện về hạ tầng công nghệ thông tin áp dụng quy định về hóa đơn, chứng từ điện tử của Nghị định này trước ngày 01 tháng 7 năm 2022.

%%%%%%%%%%%%%%%%%%%%%%%%%%%%%%%%%%%%%!

$\Rightarrow$ Theo quy định, tất cả các doanh nghiệp, tổ chức và hộ kinh doanh đều bắt buộc phải chuyển từ sử dụng hóa đơn giấy sang hóa đơn điện tử bắt đầu từ tháng 07/2022. Vì vậy, nhu cầu sử dụng và xử lý hóa đơn điện tử trở nên rất lớn. Do đó ở đồ án này, em chọn chủ đề về quản lý hóa đơn điện tử.

Đề tài xxxxxxxxxxx là một xu hướng quan trọng trong phát triển phần mềm và tiềm năng mang lại nhiều lợi ích trong việc cải thiện quá trình quản lý xxxxxxxxxxxxxxxxxxx

\end{document} % kết thúc

paragraph

% %

% % %! Aggregates/ /

% % Tổng hợp là đối tượng kinh doanh trung tâm trong Bối cảnh giới hạn của chúng ta và xác định phạm vi nhất quán trong bối cảnh giới hạn đó.

% % Tổng hợp = Mã định danh chính của Bối cảnh giới hạn của chúng ta

% \subsubsection{xxxxxxx}

% % Trong quá trình hoạt động kinh doanh, doanh nghiệp có nhu cầu chuyển đổi mô hình kinh doanh linh hoạt để có thể tồn tại và phát triển khi thị trường thay đổi. Từ đó, đáp ứng nhu cầu của khách hàng, mang lại ưu thế cạnh tranh so với các đối thủ. Do đó, các doanh nghiệp cần hệ thống chuyển đổi nhanh chóng để đáp ứng nhu cầu của mô hình kinh doanh và mong đợi của khách hàng.

Trong những năm gần đây, việc áp dụng kiến trúc vi dịch vụ ngày càng phổ biến,  đem lại nhiều lợi ích như  tách các nghiệp vụ kinh doanh   thành các dịch vụ nhỏ độc lập nhau, tắng tính   linh hoạt và khả năng chống chịu sự cố. 



tăng tính mở rộng c 
 
 khả năng mở rộng và dễ dàng phát triển và bảo trì hệ thống.



%%%%%%%%%%%%%%%%%%%%%%%%%%%%%%%%%%

% thiết kế hướng miền

$\Rightarrow$ Kiến trúc vi dịch vụ giải quyết những thách thức và hỗ trợ doanh nghiệp chuyển đổi dễ dàng. Tuy nhiên, để xây dựng được kiến trúc vi dịch vụ tốt, cần phải tạo ra các dịch vụ nhỏ phù hợp và duy trì tính độc lập. Trong đồ án này, em sử dụng thiết kế hướng miền để phân tích và xây dựng kiến trúc vi dịch vụ. Thiết kế hướng miền xác định và tổ chức các dịch vụ dựa trên việc hiểu rõ về lĩnh vực kinh doanh, giúp dự án phản ánh đúng các quy trình và quy tắc kinh doanh.

% Chủ đề đồ án

%%%%%%%%%%%%%%%%%%%%%%%%%%%%%%%%%%%%%!

Nghị định này có hiệu lực thi hành kể từ ngày 01 tháng 7 năm 2022, khuyến khích cơ quan, tổ chức, cá nhân đáp ứng điều kiện về hạ tầng công nghệ thông tin áp dụng quy định về hóa đơn, chứng từ điện tử của Nghị định này trước ngày 01 tháng 7 năm 2022.

%%%%%%%%%%%%%%%%%%%%%%%%%%%%%%%%%%%%%!

$\Rightarrow$ Theo quy định, tất cả các doanh nghiệp, tổ chức và hộ kinh doanh đều bắt buộc phải chuyển từ sử dụng hóa đơn giấy sang hóa đơn điện tử bắt đầu từ tháng 07/2022. Vì vậy, nhu cầu sử dụng và xử lý hóa đơn điện tử trở nên rất lớn. Do đó ở đồ án này, em chọn chủ đề về quản lý hóa đơn điện tử.

Đề tài xxxxxxxxxxx là một xu hướng quan trọng trong phát triển phần mềm và tiềm năng mang lại nhiều lợi ích trong việc cải thiện quá trình quản lý xxxxxxxxxxxxxxxxxxx

% \end{document} % kết thúc

% Yêu cầu nghiệp vụ của từng sub

% %

% Sơ đồ if else Đ S

% %

% sub trước model

% %

%%%%%%%%%%%%%%%%%%%%%%%%%%%%%%%%%%%%%

\end{document}

\section{xxxxxxx}

\subsection{xxxxxxx}

\subsubsection{xxxxxxx}

Trong quá trình hoạt động kinh doanh, doanh nghiệp có nhu cầu chuyển đổi mô hình kinh doanh linh hoạt để có thể tồn tại và phát triển khi thị trường thay đổi. Từ đó, đáp ứng nhu cầu của khách hàng, mang lại ưu thế cạnh tranh so với các đối thủ. Do đó, các doanh nghiệp cần hệ thống chuyển đổi nhanh chóng để đáp ứng nhu cầu của mô hình kinh doanh và mong đợi của khách hàng.

Trong những năm gần đây, việc áp dụng kiến trúc vi dịch vụ ngày càng phổ biến,  đem lại nhiều lợi ích như  tách các nghiệp vụ kinh doanh   thành các dịch vụ nhỏ độc lập nhau, tắng tính   linh hoạt và khả năng chống chịu sự cố. 



tăng tính mở rộng c 
 
 khả năng mở rộng và dễ dàng phát triển và bảo trì hệ thống.



%%%%%%%%%%%%%%%%%%%%%%%%%%%%%%%%%%

% thiết kế hướng miền

$\Rightarrow$ Kiến trúc vi dịch vụ giải quyết những thách thức và hỗ trợ doanh nghiệp chuyển đổi dễ dàng. Tuy nhiên, để xây dựng được kiến trúc vi dịch vụ tốt, cần phải tạo ra các dịch vụ nhỏ phù hợp và duy trì tính độc lập. Trong đồ án này, em sử dụng thiết kế hướng miền để phân tích và xây dựng kiến trúc vi dịch vụ. Thiết kế hướng miền xác định và tổ chức các dịch vụ dựa trên việc hiểu rõ về lĩnh vực kinh doanh, giúp dự án phản ánh đúng các quy trình và quy tắc kinh doanh.

% Chủ đề đồ án

%%%%%%%%%%%%%%%%%%%%%%%%%%%%%%%%%%%%%!

Nghị định này có hiệu lực thi hành kể từ ngày 01 tháng 7 năm 2022, khuyến khích cơ quan, tổ chức, cá nhân đáp ứng điều kiện về hạ tầng công nghệ thông tin áp dụng quy định về hóa đơn, chứng từ điện tử của Nghị định này trước ngày 01 tháng 7 năm 2022.

%%%%%%%%%%%%%%%%%%%%%%%%%%%%%%%%%%%%%!

$\Rightarrow$ Theo quy định, tất cả các doanh nghiệp, tổ chức và hộ kinh doanh đều bắt buộc phải chuyển từ sử dụng hóa đơn giấy sang hóa đơn điện tử bắt đầu từ tháng 07/2022. Vì vậy, nhu cầu sử dụng và xử lý hóa đơn điện tử trở nên rất lớn. Do đó ở đồ án này, em chọn chủ đề về quản lý hóa đơn điện tử.

Đề tài xxxxxxxxxxx là một xu hướng quan trọng trong phát triển phần mềm và tiềm năng mang lại nhiều lợi ích trong việc cải thiện quá trình quản lý xxxxxxxxxxxxxxxxxxx

% phải có CQRS (Phân chia trách nhiệm truy vấn lệnh)

CQRS là một mẫu kiến trúc riêng biệt có thể được sử dụng kết hợp với thiết kế hướng miền để đạt được những lợi ích nhất định, chẳng hạn như cải thiện hiệu suất và khả năng mở rộng. Tuy nhiên, nó không phải là một yêu cầu để triển khai thiết kế hướng miền.

% phải có event

Ngôn ngữ chung (Ubiquitous Language)

%%%%%%%%%%%%%%%%%%%%%%%%%%%%%%%%%%%%%

\end{document} % kết thúc

Cách tiếp cận này nhấn mạnh tính mô - đun, tính linh hoạt và khả năng phục hồi, cho phép các nhóm làm việc đồng thời trên các phần khác nhau của hệ thống và cho phép phát hành nhanh hơn và thường xuyên hơn. Các vi dịch vụ thường dựa vào các giao thức truyền thông nhẹ, chẳng hạn như REST và thường được triển khai bằng các công nghệ chứa trong bộ chứa như Docker và Kubernetes.

\subsubsection{DevOps Ứng dụng, áp dụng, liên quan,....}

\subsubsection{CI/CD}

\subsubsection{Docker}

\subsubsection{Kubernetes}

%@ Tất cả phải dùng ulli