Trong kiến trúc vi dịch vụ, các dịch vụ phải tương tác quan hệ với nhau, dẫn đến sự xuất hiện của mối quan hệ phụ thuộc. Những mối quan hệ này cần được quản lý chặt chẽ để hoạt động độc lập, nhất quán và linh hoạt. Do đó, cần phải ghi lại các mối quan hệ thông qua việc sử dụng bản đồ bối cảnh. \emph{Bản đồ bối cảnh (Context Maps)} là sự thể hiện trực quan của hệ thống, thể hiện các thành phần, liên kết và mối quan hệ.

\end{document}

% %!<! - - Context Mapping : https:// thiết kế hướng miền - practitioners.com/context - map - - >


Bản đồ bối cảnh

Trong Thiết kế hướng miền, Bản đồ bối cảnh là một công cụ trực quan được sử dụng để hiển thị mối quan hệ giữa các bối cảnh giới hạn khác nhau trong một miền. Về cơ bản, nó là một bản đồ hiển thị cách các bối cảnh khác nhau trong một miền tương tác với nhau, bao gồm ranh giới, sự phụ thuộc và kiểu giao tiếp của chúng.

Bối cảnh giới hạn rất quan trọng trong thiết kế hướng miền vì chúng giúp xác định và quản lý độ phức tạp của các miền lớn bằng cách chia chúng thành các phần nhỏ hơn, dễ quản lý hơn. Bằng cách hiểu mối quan hệ giữa các bối cảnh khác nhau, các nhóm phát triển có thể cộng tác hiệu quả hơn và đảm bảo rằng mỗi bối cảnh được thiết kế theo cách phù hợp với mục tiêu chung của miền. Bản đồ bối cảnh cũng giúp làm nổi bật các khu vực có xung đột hoặc chồng chéo tiềm ẩn giữa các bối cảnh, có thể được giải quyết sớm để tránh các vấn đề về sau.

Bản đồ ngữ cảnh thường bao gồm các thành phần sau: 
Bản đồ bối cảnh cung cấp chế độ xem cấp cao về miền và cách các bối cảnh giới hạn khác nhau có liên quan với nhau, giúp các bên liên quan xác định kiến trúc, sự phụ thuộc và rủi ro của hệ thống.
  


% %!<! - - Bounded Context: https:// thiết kế hướng miền - practitioners.com/home/glossary/bounded - context - - >

% bối cảnh giới hạn là ranh giới trong đó một mô hình miền cụ thể tồn tại và hợp lệ. Các ngữ cảnh giới hạn giúp xác định phạm vi của mô hình miền và thiết lập sự hiểu biết rõ ràng về ngôn ngữ được sử dụng trong ngữ cảnh đó.

% Các bối cảnh giới hạn phải độc lập trong bối cảnh riêng của chúng, nhưng chúng vẫn có thể cần tương tác với các bối cảnh giới hạn khác để hoàn thành trách nhiệm của chính chúng. Mặc dù chúng độc lập về mặt mô hình nhưng chúng có thể cần giao tiếp với các bối cảnh giới hạn khác để trao đổi thông tin hoặc cộng tác trong một số nhiệm vụ nhất định. Vì vậy các bối cảnh giới hạn có thể có mối quan hệ với nhau. Những mối quan hệ này rất quan trọng vì chúng giúp xác định sự tương tác giữa các phần khác nhau của hệ thống, cũng như thiết lập ranh giới và trách nhiệm giữa các nhóm khác nhau làm việc trên cùng một hệ thống. Có một số loại mối quan hệ bối cảnh giới hạn, bao gồm:

% Quan hệ đối tác : Đây là mối quan hệ trong đó hai hoặc nhiều bối cảnh giới hạn cộng tác và chia sẻ thông tin. Mối quan hệ hợp tác có thể đạt được thông qua một sự kiện tích hợp hoặc bằng cách gọi một dịch vụ.

% Hạt nhân được chia sẻ : Trong mối quan hệ này, hai bối cảnh giới hạn chia sẻ một lược đồ mô hình, mã hoặc CSDL chung. Các thành phần được chia sẻ phải ổn định, hoàn thiện và được cả hai nhóm đồng ý. Những thay đổi được thực hiện đối với các thành phần dùng chung cần phải được phối hợp cẩn thận để đảm bảo chúng không phá vỡ chức năng của bối cảnh khác.

% Khách hàng - Nhà cung cấp : Đây là mối quan hệ trong đó một bối cảnh giới hạn cung cấp dịch vụ hoặc dữ liệu cho bối cảnh khác. Bối cảnh khách hàng dựa vào bối cảnh nhà cung cấp để có những khả năng hoặc dữ liệu nhất định. Những thay đổi trong bối cảnh nhà cung cấp có thể tác động đến bối cảnh khách hàng, vì vậy điều quan trọng là phải quản lý mối quan hệ này một cách cẩn thận.

% Người theo chủ nghĩa tuân thủ : Mối quan hệ này tồn tại khi một bối cảnh giới hạn tuân theo cùng một ngôn ngữ và khái niệm phổ biến như một bối cảnh giới hạn khác. Điều này đảm bảo rằng giao tiếp giữa hai bối cảnh là rõ ràng và rõ ràng.

% Lớp chống đổ vỡ : Đây là mối quan hệ trong đó ngữ cảnh được giới hạn sử dụng một lớp để dịch giữa ngôn ngữ của chính nó và ngôn ngữ của ngữ cảnh giới hạn khác. Điều này cho phép hai ngữ cảnh giao tiếp với nhau ngay cả khi chúng có từ vựng hoặc mô hình khác nhau.

% Dịch vụ máy chủ mở : Mối quan hệ này xảy ra khi một bối cảnh giới hạn hiển thị một API công khai, mở mà các ngữ cảnh khác có thể sử dụng. Điều này cho phép các ngữ cảnh khác tận dụng chức năng của ngữ cảnh máy chủ theo cách được tiêu chuẩn hóa.

% Ngôn ngữ được xuất bản : Trong mối quan hệ này, một ngữ cảnh giới hạn sẽ xuất bản từ vựng và mô hình của nó cho các ngữ cảnh khác sử dụng. Điều này hữu ích khi nhiều bối cảnh cần cộng tác nhưng không muốn tích hợp trực tiếp. Ngôn ngữ được xuất bản đảm bảo rằng ý nghĩa của các thuật ngữ nhất quán trong mọi ngữ cảnh.

% Các cách riêng biệt : Mối quan hệ này đề cập đến tình huống trong đó hai hoặc nhiều bối cảnh giới hạn không còn có mối quan hệ với nhau và các nhóm chịu trách nhiệm về chúng chọn làm việc độc lập.

% %!<! - - Bounded Context Relationships : https:// thiết kế hướng miền - practitioners.com/bounded - context - relationship - - >

% %!<! - - Bounded Context Relationships : https:// thiết kế hướng miền - practitioners.com/bounded - context - relationship - - >

<!--@============================================== -->