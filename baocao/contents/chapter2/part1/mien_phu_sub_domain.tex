Một miền lớn được tạo thành từ nhiều \emph{miền phụ (Sub - Domain)}. Trong thực tế, một miền kinh doanh phức tạp không thể có một chuyên gia ngành có kiến thức về tất cả các miền phụ.

\begin{example} Trong miền thương mại điện tử lớn có thể có một số miền phụ sau:

\begin{itemize}

\item \textbf{Miền phụ quản lý hàng tồn kho:} liên quan đến việc quản lý sản phẩm trong kho hàng.

\item \textbf{Miền phụ quản lý khách hàng:} liên quan đến việc quản lý tài khoản khách hàng.

\item \textbf{Miền phụ vận chuyển:} liên quan đến việc quản lý việc vận chuyển giao hàng.

\end{itemize}

\end{example}

% @Slide thì Phân loại các miền phụ

Trong thiết kế hướng miền, có ba loại miền phụ là:

\begin{itemize}

\item Miền phụ chung (Generic Subdomain)

\item Miền phụ cốt lõi (Core Subdomain)

\item Miền phụ hỗ trợ (Supporting Subdomain)

\end{itemize}