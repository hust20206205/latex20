%%%%%%%%%%%%%%%%%%%%%%%%%%%%%%%%%%%%%!

Hóa đơn điện tử là hóa đơn có mã hoặc không có mã của cơ quan thuế được thể hiện ở dạng dữ liệu điện tử do tổ chức, cá nhân bán hàng hóa, cung cấp dịch vụ lập bằng phương tiện điện tử để ghi nhận thông tin bán hàng hóa, cung cấp dịch vụ theo quy định của pháp luật về kế toán, pháp luật về thuế, bao gồm cả trường hợp hóa đơn được khởi tạo từ máy tính tiền có kết nối chuyển dữ liệu điện tử với cơ quan thuế, trong đó:

a. Hóa đơn điện tử có mã của cơ quan thuế là hóa đơn điện tử được cơ quan thuế cấp mã trước khi tổ chức, cá nhân bán hàng hóa, cung cấp dịch vụ gửi cho người mua. Mã của cơ quan thuế trên hóa đơn điện tử bao gồm số giao dịch là một dãy số duy nhất do hệ thống của cơ quan thuế tạo ra và một chuỗi ký tự được cơ quan thuế mã hóa dựa trên thông tin của người bán lập trên hóa đơn.

b. Hóa đơn điện tử không có mã của cơ quan thuế là hóa đơn điện tử do tổ chức bán hàng hóa, cung cấp dịch vụ gửi cho người mua không có mã của cơ quan thuế.

%%%%%%%%%%%%%%%%%%%%%%%%%%%%%%%%%%%%%!

