Trong quá trình hoạt động kinh doanh,   doanh nghiệp  có nhu cầu chuyển đổi  mô hình kinh doanh   linh hoạt  để có thể tồn tại và phát triển khi thị trường thay đổi.  Từ đó, đáp ứng nhu cầu của khách hàng, mang lại ưu thế cạnh tranh so với các đối thủ. Do đó, các doanh nghiệp cần hệ thống chuyển đổi nhanh chóng để đáp ứng nhu cầu của mô hình kinh doanh và mong đợi của khách hàng.




% kiến trúc vi dịch vụ


Trong những năm gần đây, việc áp dụng kiến trúc vi dịch vụ  ngày càng phổ biến.

Sự ra đời của kiến trúc vi dịch vụ đem lại nhiều lợi ích như làm giảm độ phức tạp trong việc xây dựng các ứng dụng lớn cho doanh nghiệp, tăng tính mở rộng cũng như nâng cao khả năng bảo trì hệ thống.

% Sử dụng kiến trúc vi dịch vụ trong hệ thống quản lý xxxxxxxxxxxxxxxx có thể giúp tách các chức năng khác nhau thành các dịch vụ nhỏ độc lập nhau, giúp cải thiện tính linh hoạt, khả năng mở rộng và dễ dàng phát triển và bảo trì hệ thống.

% Kiến trúc vi dịch vụ cho phép các kiến trúc vi dịch vụ hoạt động độc lập linh hoạt và đáng tin cậy trong việc quản lý người dùng, quản lý sản phẩm, quản lý đơn hàng và thanh toán.

%%%%%%%%%%%%%%%%%%%%%%%%%%%%%%%%%%
% thiết kế hướng miền


$\Rightarrow$ Kiến trúc vi dịch vụ giải quyết những thách thức và hỗ trợ doanh nghiệp chuyển đổi dễ dàng. Tuy nhiên, để xây dựng được kiến trúc vi dịch vụ tốt, cần phải tạo ra các dịch vụ nhỏ phù hợp và duy trì tính độc lập. Trong đồ án này, em sử dụng thiết kế hướng miền để phân tích và xây dựng kiến trúc vi dịch vụ. Thiết kế hướng miền xác định và tổ chức các dịch vụ dựa trên việc hiểu rõ về lĩnh vực kinh doanh, giúp dự án phản ánh đúng các quy trình và quy tắc kinh doanh.


% Chủ đề đồ án

%%%%%%%%%%%%%%%%%%%%%%%%%%%%%%%%%%%%%!

Nghị định này có hiệu lực thi hành kể từ ngày 01 tháng 7 năm 2022, khuyến khích cơ quan, tổ chức, cá nhân đáp ứng điều kiện về hạ tầng công nghệ thông tin áp dụng quy định về hóa đơn, chứng từ điện tử của Nghị định này trước ngày 01 tháng 7 năm 2022.

%%%%%%%%%%%%%%%%%%%%%%%%%%%%%%%%%%%%%!


$\Rightarrow$ Theo quy định, tất cả các doanh nghiệp, tổ chức và hộ kinh doanh đều bắt buộc phải chuyển từ sử dụng hóa đơn giấy sang hóa đơn điện tử bắt đầu từ tháng 07/2022. Vì vậy, nhu cầu sử dụng và xử lý hóa đơn điện tử trở nên rất lớn. Do đó ở đồ án này, em chọn chủ đề về quản lý hóa đơn điện tử.

Đề tài xxxxxxxxxxx là một xu hướng quan trọng trong phát triển phần mềm và tiềm năng mang lại nhiều lợi ích trong việc cải thiện quá trình quản lý xxxxxxxxxxxxxxxxxxx